\documentclass[11pt,letterpaper]{article}
\usepackage[margin=.75in]{geometry}
\usepackage{amsmath}
\usepackage{graphicx}
\usepackage{amssymb} \usepackage{natbib}
\usepackage{float} \usepackage{appendix}
\usepackage{hyperref}
\usepackage{mathrsfs}
\floatstyle{ruled} \restylefloat{table} \restylefloat{figure}
\bibliographystyle{unsrtnat}

\newcommand{\floatintro}[1]{
  
  \vspace*{0.1in}
  
  {\footnotesize

    #1
    
  }
  
  \vspace*{0.1in} }
\newcommand{\Hline}{\noindent\rule{17cm}{0.5pt}} \title{Homework 1:
  Labor Economics} \author{Dhananjay Ghei} \date{December 3, 2018}
\begin{document}
\maketitle
\section{Preliminary analysis}
\begin{enumerate}
\item Visit the BHPS website and familiarise yourself with the basic
  structure and contents of the BHPS data. What features make it a
  suitable data set for the estimation of the BM model? \\ \Hline \\
\item Open the file and answer the following questions:
  \begin{enumerate}
  \item What is the sample size? What is the sex ratio in the sample?
  \item What is the sample unemployment rate? What is the sample
    unemployment rate of men? Of women? Or workers in each education
    category? 
  \item What proportion of initial spells are right-censored? Answer
    the same question for each type of first spell (job or
    unemployment spell).
  \end{enumerate}
\Hline \\
\item Construct the initial (spell-1) cross-sectional CDF (\texttt{G})
  and density of log wages \texttt{logw1}. Produce the plots of these
  two objects.\\ \Hline \\
\item Create a variable categorising \texttt{logw1} into 25 bins (ie,
  percentiles 1-4, 5-8,9-12,\dots,97-100) and a variable containing
  the mean spell-1 duration (\texttt{spelldur1}) within each of these
  25 bins. Plot those mean durations against the wage percentiles. Is
  this consistent with the BM model?\\ \Hline \\
\item Explain how one can obtain a non-parametric estimate of the wage
  sampling distribution $F$ from the data. Construct this
  non-parametric estimate, and plot it on the same graph as
  \texttt{G}. Is this consistent with the theory? What else can you
  say about the estimate of $F$? \\ \Hline \\
\end{enumerate}
\section{Estimation}
\begin{enumerate}
\item Write code for the MLE estimation of the BM model following the
  two-step protocol of Bontemps, Robin and Van den Berg (1999). \\ \Hline \\
\item Write code for computing the standard errors of the estimates
  $\delta$, $\lambda_0$ and $\lambda_1$, explaining the assumptions
  upon which those standard errors rely. \\ \Hline \\
\item The file \texttt{BM\_data\_simulated.csv} contains artificial data
  resulting from a simulation of 5,000 workers behaving according to
  the BM model with parameters $\delta=0.01$, $\lambda_0=0.1$,
  $\lambda_1=0.05$ (monthly values). Run your ML estimation routine on
  the simulated data, and check that your estimates against the true
  parameter values. \\ \Hline \\
\end{enumerate}
\section{Playing around with the model}
\begin{enumerate}
\item What is the predicted unemployment rate from the estimates
  obtained in Section II? Compare it with the sample unemployment
  rate, and discuss the possible reasons for any discrepancy. \\ \Hline \\
\item Construct kernel density estimates of the cross-section
  distribution of wages $g(w)$ and of the sampling distribution
  $f(w)$. Plot both densities on the same graph. \\ \Hline \\
\item Construct the distribution of firm productivity that
  rationalises the observed wage distribution within the BM
  model. Plot firm productivity against wages, and against the
  cross-section CDF of wages $G(w)$. Do you notice anything wrong? \\ \Hline \\
\item Looking at the predicted profit rate of high-productivity firms,
  what else can you say about the BM model? \\ \Hline \\
\end{enumerate}
\end{document}
